\chapter{Blinn-Phong Lighting}

Thanks to all the previous chapters, we now have enough knowledge to draw
something interesting on the screen.
We start off by setting up the scene that will be rendered.
Then, for the rest of the chapter, we work on adding lighting to said scene.
We will see how adding lighting to a scene can really make the difference
for our visuals.

\section{Rendering A Simple Scene}

Our scene consists of three objects.
A floor, a cube, and a light.

The floor is a simple quad.
It is positioned in the origin of our scene, which is $(0, 0, 0)$.
It has no rotation.
It is scaled on all axis by a factor of $10$.
And, for now, we give it an rgb color value of $(0.5, 0.5, 0.5)$,
which is gray.

The cube is positioned in the origin of our scene, placed on the floor.
We rotate it by $45$ degrees around the vertical $z$ axis.

Our floor will be a simple quad.
Our light will also be represented by a cube.


\subsection{}

\section{Colors}

\section{Adding Lighting To Our Scene}

\section{Ambient Lighting}

\section{Diffuse Lighting}

\section{Specular Lighting}

\section{Materials}

\section{Light Properties}
