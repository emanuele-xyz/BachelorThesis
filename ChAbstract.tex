\chapter*{Abstract}

The aim of this work is to explore the Vulkan graphics API.
We do this by studying how the API works and using it to produce a working
application.
Each chapter tries to focus on one single Vulkan concept producing a demo
that shows how it works in practice.

In \autoref{chap:Vulkan} we introduce the history behind Vulkan.
This lets us understand the reasons that guided the design of the API.
In \autoref{chap:InitializingVulkan} we explain all the work that is almost
always necessary to get a Vulkan application up and running.
In \autoref{chap:ClearWindow} we finally do the simplest form of rendering:
we clear the window with a flat color.
This may seem trivial, but it's an important stepping stone.
Doing this, we can see all the concepts that come into play when drawing something
on the screen.
In \autoref{chap:Triangle} we finally write the computer graphics hello world
program: rendering a triangle.
In \autoref{chap:Vertices} and in \autoref{chap:Uniforms} we see how to send
data from the CPU to the GPU, making our application more flexible.
In \autoref{chap:DepthTesting} we see how to solve a common problem in computer
graphics: how to determine the order in which to draw objects, so that closer
objects cover far away ones.
In \autoref{chap:Scene} we take a break from Vulkan and
explain a very simple way to describe objects we want to draw, and how
to place them inside a virtual world.
In \autoref{chap:BlinnPhong} we use all the concepts we have learned so far
to implement a lighting model.
In \autoref{chap:MSAA} we improve the visual quality of our application
enabling a Vulkan feature called multisample anti aliasing.
