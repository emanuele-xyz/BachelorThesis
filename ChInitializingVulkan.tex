\chapter{Initializing Vulkan}

\section{VkInstanceCreateInfo}

To access any of the functionalities offered by Vulkan we first have to create a Vulkan instance.
To do this we call vkCreateInstance.
When calling this function we need to pass a pointer to a VkInstanceCreateInfo struct.
This struct collects all the information needed to configure our Vulkan instance.

\begin{minipage}{\linewidth}{\noindent}
\lstinputlisting[
    language=C++,
    caption={VkInstanceCreateInfo initialization},
    label={lst::VkInstanceCreateInfo}
    ]{src/VkInstanceCreateInfo.cpp}
\end{minipage}

\subsection{VkApplicationInfo}

We can see that the VkInstanceCreateInfo struct is not the only thing we need.
We have to specify a pointer to a VkApplicationInfo struct. Such struct describes
our Vulkan application.

\begin{minipage}{\linewidth}{\noindent}
    \lstinputlisting[
        language=C++,
        caption={VkApplicationInfo initialization},
        label={lst::VkApplicationInfo}
        ]{src/VkApplicationInfo.cpp}
\end{minipage}

\subsection{Layers}

Vulkan is designed to be very lean and with as little overhead as possible.
This


\subsection{Extensions}
