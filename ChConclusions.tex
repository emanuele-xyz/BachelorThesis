\chapter*{Conclusion}

The goal of this work was to explore the Vulkan graphics API, studying
its ideas and seeing how to put them into practice.
During this work, we have dealt with many concepts that, not only, are core to
Vulkan, but also to other modern graphics APIs, such as Direct3D and Metal.
Having a good grasp on these concepts allows us to be more at ease
transitioning to these other APIs.
Having a good understanding of Vulkan, we can also implement many real
time rendering ideas: we have seen an example using Blinn-Phong lighting.

We have met a lot of Vulkan concepts so far, but there are many more
that we haven't faced, being more advanced and specific.
We haven't discussed on how to use multiple render pass subpasses and how
to describe the dependencies between them.
We haven't talked about how we could render multiple frames concurrently, using
a pool of command buffers.
We haven't faced the problem of GPU memory allocation and how to implement
an appropriate memory allocator.
We haven't seen how to deal with textures, generating mipmaps and mapping them
to 3D objects.
These are some ideas that I would like to suggest to people that want to
keep delving deeper into Vulkan.
