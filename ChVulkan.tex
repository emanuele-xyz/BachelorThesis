\chapter{Vulkan}

\section{What is Vulkan?}

\begin{wrapfigure}{l}{0.4\textwidth}
    \begin{center}
        \includegraphics[scale=0.10]{images/ChVulkan/VulkanLogo.png}
    \end{center}
    \caption{Vulkan logo}
    \label{fig:VulkanLogo}
\end{wrapfigure}

Vulkan is a modern graphics API. It is maintained by the Khronos Group.
Vulkan is meant to abstract how modern GPUs work.
Using Vulkan, the programmer can write more performant code.
The better performance comes at the cost of having a more verbose and low level API compared to
other existing APIs such as OpenGL or Direct3D 11 and prior.
Vulkan is not the only modern graphics API, other such APIs are Direct3D 12 and Metal.
Nonetheless, Vulkan has the advantage of being fully cross platform.

\section{What problems does Vulkan solve?}

\begin{wrapfigure}{l}{0.4\textwidth}
    \begin{center}
        \includegraphics[scale=0.10]{images/ChVulkan/OpenGLLogo.png}
    \end{center}
    \caption{OpenGL logo}
    \label{fig:OpenGLLogo}
\end{wrapfigure}

Common graphics APIs like OpenGL or Direct3D were developed during the 1990s.
At that time, graphics card hardware was very limited not only in terms of computational
power but also from a functionality standpoint. As time progressed, graphics card architectures
continued to evolve, offering new functionalities.
All these new functionalities had to be integrated with the old existing APIs.
The more new functionalities were integrated, the more the GPU's driver complexity grew.
Such complicated GPU drivers are inefficient and are also the cause of many
inconsistencies between implementations of the same graphics API but on different GPUs.

\section{How does Vulkan solve these problems?}

Vulkan doesn't suffer from the problems we saw above because it has been designed from scratch
and with modern GPU's architecture in mind.
It reduces the driver overhead by being more verbose and low level.
It is also designed to be multithreaded, allowing the programmer to submit GPU commands from
different threads.
This is very beneficial to performance, since modern CPUs usually have more than one core.
