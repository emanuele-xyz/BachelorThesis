\chapter{Vulkan}

\section{OpenGLBook Notes}

On the most fundamental level, OpenGL (graphics APIs in general) is a software
interface that allows a programmer to communicate with graphics hardware.

The first Personal Computers started to appear in the mid to late 1970s,
but were regarded as enthusiast machines for hobbyists.
This changed somewhat with the release of the Apple II by Apple Computer in
1977, and the PET by Commodore International. These machines popularized the
concept of computers for the home, but did not have too much to offer in terms
of computer graphics.

This somewhat changed in the 1980s when technologies such as GUI were introduced
to the personal computing market. The first dedicated graphics add-on cards also
started to appear during this era, notably the CGA (Color Graphics Adapter)
by IBM was the first color graphics card for the IBM PC platform, which
would pave the way for future developments by standardizing a method of drawing
computer graphics but didn't offer much in terms of graphics capabilities.

During the 1970s and 1980s, most video games ran on specialized systems,
movies made use of computer animation only sparsely, and real-time 3D
graphics were for visualization purposes only since there was no consumer
hardware that was fast enough.
These years were known for their many firsts on consumer hardware, but
it wasn't until the late 1980s to the early 1990s when computer games took
a strong hold on the PC platform and a real push for better looking and better
performing real-time graphics began.

Silicon Graphics (commonly referred to as SGI) was a company founded in
1981 that specialized in 3D computer graphics and developed software and hardware
specifically for this purpose. One software library that SGI developed was
IRIS GL (Integrated Raster Imaging System Graphical Library) used for generating
2D and 3D graphics on SGI's high performance workstations.

In the early 1990s, SGI was the market leader in 3D graphics workstations
because of their high performance hardware and easy to use software.
IRIS GL was the de facto industry standard 3D graphics library, overshadowing
all other developments and attempts to standardize a 3D graphics interface.
But despite its popularity, IRIS GL had one major problem: it was a proprietary
system fused to SGI's own platforms, and competitors were closing in on SGI's
advantage with their own APIs.

In a bold move, SGI cleaned up IRIS GL, removed all functionality that
did not relate to computer graphics and released it to the public in 1992 as
OpenGL (Open Graphics Library), a cross-platform standardized API for real-time
computer graphics.

Software vendors would have to provide their own implementations of the
OpenGL standard on their platforms, and hardware vendors programs that
allowed OpenGL to talk to the underlying graphics hardware called "device drivers".

The name "OpenGL" was not just chosen because it sounded like a fine buzzword,
it also contains some actual meaning. Since OpenGL is an evolving specification,
someone has to decide what goes in it. So in 1992, the ARB (OpenGL Architecture
Review Board) was founded, which comprised of several high profile software and
hardware vendors who collectively decided the future of the OpenGL standard through
a voting system. Besides determining what new features went into the OpenGL
specification, it also decided which extensions would be promoted to become
core features of the next OpenGL release.

OpenGL quickly became the industry leading real-time graphics API,
as it was basically the only one available on multiple platforms.

In the late 1990s, OpenGL established itself as an industry standard for
3D computer graphics, not just for CAD programs, although it was the only
contender in that market. PC video games such as Quake 2, Unreal, and Half-Life
took full advantage of OpenGL to show off their full potential and were widely
popular. Around this time, the first consumer-grade dedicated 3D graphics
hardware started to appear, changing the video game industry forever.

During the early 2000s, GPU performance grew exponentially as more software
features were moved to the GPU. The CPU became obsolete for rendering real-time
3D graphics since it could not keep up with GPU developments. In fact, the current
method of rendering 3D graphics saw the CPU as such major bottleneck that new
methods had to be invented to circumvent its use.

In rendering real-time computer graphics, the software pipeline exists to
describe what we'd like to see on the screen.
For example, if we'd like to display a green square on the screen,
computer software would describe with which dimensions, color, and at which
position of the screen to draw the square.

The software pipeline also provides access to functionality that draws
the geometry onto the screen. It is worth noting that the software pipeline
does not actually do any drawing or transformations, since on modern systems
this functionality is entirely implemented by the hardware.

The software pipeline consists of several different layers, each with their
own very specific purposes.

The first is the Application layer, which is your program, the program that
invokes drawing commands. The application serves as a controller of the overall
process and oversees all of the user-level operations such as creating windows,
threads, memory allocation, complex user data-types, and making calls to external
libraries such as OpenGL or Direct3D through their respective interfaces.

The next layer is the Abstraction layer, which contains the OpenGL or Direct3D
API implementations.
The Abstraction layer serves as a dispatch to the next layer by implementing
hardware-level functionality in a usable and standardized format.

The abstraction layer passes its commands to the Device Driver, a software
communication layer to the hardware. This layer is entirely invisible to the
developer, since it cannot be interacted with through your program.
The device driver interprets the commands passed to it by the Abstraction
layer and relays them to the underlying device in a format that the hardware
can understand and easily process.

\section{Learn WebGL}

In the early days, if you wanted to create computer graphics you had to
write programs that directly talked to the graphics hardware. As new hardware
designs were created, the software had to be completely re-written to work with
the new hardware.

The premiere computer graphics company in the late 1980's and throughout
the 1990's was Silicon Graphics, Inc. They were a hardware company, but
they shipped their computers with a proprietary computer graphics application
programmer interface (API) known as IRIS Graphics Language (IRIS GL).

\section{Graphics Book}

In the first desktop computers, the contents of the screen were managed
directly by the CPU. For example, to draw a line segment on the screen, the CPU
would run a loop to set the color of each pixel that lies along the line.
Needless to say, graphics could take up a lot of the CPU's time. And graphics
performance was very slow, compared to what we expect today. So what has changed?
Computers are much faster in general, of course, but the big change is that in
modern computers, graphics processing is done by a specialized component called
a GPU, or Graphics Processing Unit. A GPU includes processors for doing graphics
computations; in fact, it can include a large number of such processors that work
in parallel to greatly speed up graphical operations. It also includes its own
dedicated memory for storing things like images and lists of coordinates.
GPU processors have very fast access to data that is stored in GPU memory—much
faster than their access to data stored in the computer's main memory.

To draw a line or perform some other graphical operation, the CPU simply
has to send commands, along with any necessary data, to the GPU, which
is responsible for actually carrying out those commands. The CPU offloads most
of the graphical work to the GPU, which is optimized to carry out that work very
quickly. The set of commands that the GPU understands make up the API of the GPU.
OpenGL is an example of a graphics API, and most GPUs support OpenGL in the sense
that they can understand OpenGL commands, or at least that OpenGL commands can
efficiently be translated into commands that the GPU can understand.

\section{History of Computer Graphics}

Computer Graphics (CG) was first developed as a visualization tool.
Computer graphics were basically introduced for scientists and engineers
in government and corporate research centers.

\section{In a Nutshell: History of OpenGL}

Application programming interfaces define how applications interact with
the hardware components of a computer. They define a set of subroutines,
protocols and tools for building application software. The main goal of an
API is to abstract the underlying implementation of software and hardware and
provide the developer the tools to interact with them. Graphics APIs are
implemented by driver software that is written by graphics hardware vendors
specifically for each hardware. They are the medium with which developers can
control the dedicated graphics hardware by issuing various
implementation-agnostic, higher-level commands to the software driver
who in turn translates them to lower level commands for each specific
hardware. The next sections of this post will talk about OpenGL,
the group that created it, its history and its design philosophies.

OpenGL has been developed for over 26 years and that is a significant
amount of time in the development of cutting-edge technology.
Back in 1992 the best Intel CPU was the 80486 and math coprocessors
were an optional component. Apple computers were using Motorola
68K-derived processors and PowerPC processors were not available until
the second half of 1992. Home personal computers did not have any kind of
high-performance hardware graphics acceleration. The only way to access this
rare luxury and be able to use OpenGL was through an expensive workstation.
The state of the art in graphics was using software rendering. The best that
could be achieved without hardware acceleration was a small amount of simple
colored, filled polygons.

As time passed, the prices of graphics hardware declined, and their
performance increased. Furthermore, new features were added to the now
low-cost graphics processors which in turn were added to the OpenGL specifications
as well. Those extra features were initially added to OpenGL as extensions
which were proposed by members of the OpenGL ARB. Some of these extensions
interacted well with each other and with already existing feature of OpenGL
and some did not. Moreover, as newer ways of taking advantage of the GPU were
invented they were added in OpenGL as well, resulting in having multiple ways
available to do the same thing.

\section{Red Book}

In the very old days, drawing graphics was slow. The only way to draw graphics
quickly was to get a super computer. The early special effects companies
went bankrupt because they tried to buy Crays and things like that.
In the late 1970s and early 1980s, workstations started to be produced
commercially.

In the late 1970s, a professor at Stanford (Jim Clark) had the idea to
try to build specialized hardware to do graphics (relatively) fast and fit
it into a workstation. The plan was based on the existence of special chips
for performing certain mathematical operations (like a chip for doing 4x4
matrix multiplies). He and his students started a company to bring the
machines to market. The company was called Silicon Graphics (SGI).

SGI was really successful in the late 1980s and early-mid 1990s.
They dominated the graphics hardware market. Most interesting graphics software
ran on their hardware. It was the machine you wanted to have if you were doing
graphics.

In the mid- to late- 1990s, other companies realized that GL was a good
way to program graphics, and wanted to let programmers work that way.
So, a standards group was formed and they created OpenGL. OpenGL was designed
so that it didn't require SGIs hardware and operating system. But, it worked
like GL, which was designed with SGI's hardware in mind. Mostly this was a
good thing (since the hardware was based on good ways to think about graphics).
But some things were weird, because the hardware did things in a
weird way (like lighting). And some features were really historical: they were
designed to support what early 1990s SGI hardware did, and the best way to use that.

For a while, this was great. We had a convenient way to program graphics
that worked across different platforms.
But then, graphics hardware evolved. Very quickly. OpenGL had to be extended
to provide people with access to these new features.

But worse, while the old ways of programming graphics (i.e. what the SGI
hardware did in 1990) was convenient for many things, and was the most efficient
way to do stuff for circa 1990 computers, it was not at all how modern hardware
worked. In fact, it was quite inefficient since it was based on assumptions
about what is fast and slow that are very different than the are now.

So, in the late 2000s, early 2010s, the folks in charge of OpenGL decided to
remove all of the old circa 1980s/1990s stuff designed for SGI hardware.
They call this part of OpenGL "legacy" OpenGL.

\section{What is Vulkan?}

\begin{wrapfigure}{l}{0.4\textwidth}
    \begin{center}
        \includegraphics[scale=0.10]{images/ChVulkan/VulkanLogo.png}
    \end{center}
    \caption{Vulkan logo}
    \label{fig:VulkanLogo}
\end{wrapfigure}

Vulkan is a modern graphics API. It is maintained by the Khronos Group.
Vulkan is meant to abstract how modern GPUs work.
Using Vulkan, the programmer can write more performant code.
The better performance comes at the cost of having a more verbose and low level API compared to
other existing APIs such as OpenGL or Direct3D 11 and prior.
Vulkan is not the only modern graphics API, other such APIs are Direct3D 12 and Metal.
Nonetheless, Vulkan has the advantage of being fully cross platform.

\section{What problems does Vulkan solve?}

\begin{wrapfigure}{l}{0.4\textwidth}
    \begin{center}
        \includegraphics[scale=0.10]{images/ChVulkan/OpenGLLogo.png}
    \end{center}
    \caption{OpenGL logo}
    \label{fig:OpenGLLogo}
\end{wrapfigure}

Common graphics APIs like OpenGL or Direct3D were developed during the 1990s.
At that time, graphics card hardware was very limited not only in terms of computational
power but also from a functionality standpoint. As time progressed, graphics card architectures
continued to evolve, offering new functionalities.
All these new functionalities had to be integrated with the old existing APIs.
The more new functionalities were integrated, the more the GPU's driver complexity grew.
Such complicated GPU drivers are inefficient and are also the cause of many
inconsistencies between implementations of the same graphics API but on different GPUs.

\section{How does Vulkan solve these problems?}

Vulkan doesn't suffer from the problems we saw above because it has been designed from scratch
and with modern GPU's architecture in mind.
It reduces the driver overhead by being more verbose and low level.
It is also designed to be multithreaded, allowing the programmer to submit GPU commands from
different threads.
This is very beneficial to performance, since modern CPUs usually have more than one core.
